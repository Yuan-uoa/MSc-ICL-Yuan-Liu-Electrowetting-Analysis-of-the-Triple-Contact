\chapter{Conclusion}
\hspace{0em}\indent The study employs a method that combines modelling and conformal mapping to derive the complex potential, the surface curve, and the surface charge density, of charged droplets. By omitting the detailed local geometry, this approach simplifies the physical problem, allowing for a more tractable and generalizable droplet curve solution. The results are consistent with experimental observations, supporting the method's validity to a sufficiently accurate scale. 

However, the limitations of the method must also be acknowledged. In the local contact angle region, the current mapping function proves inadequate, and the global complex potential function fails to accurately capture the local contact angle.

\vspace{1em}\\
\noindent \textbf{Key findings include}:
\begin{itemize}
    \item The boundary functions of charged droplets are derived via multiple methods, revealing that the Crowdy map contains analytical solutions for the $90\degree$ contact angle charged droplets on a conducting substrate, where an induced charge is located above the droplet.

    \item Complex potential for the charge, dielectric, and $0\degree$ contact angle droplet model were derived. Conformal mapping techniques were used to study the corresponding thin-film models and positions of induced charges in a dielectric with specific symmetry. The shielding effect of the dielectric was analysed. 

    \item The local droplet height function, based on the surface charge density, is formulated from the complex potential function. The droplet curves for two scenarios were derived, and the surface charge density and droplet curves show agreement with physical and numerical studies.

\end{itemize}

\noindent\textbf{Future Work}
\begin{itemize}
    \item The electrical, mechanical, and fluid properties near the contact angle are more likely local experimental or numerical issues. Achieving a complex potential function that is applicable both globally and locally through conformal mapping presents significant challenges. Therefore, the contact angle should be treated as an exogenous, experimentally determined physical property. Methods such as the generalised Joukowski transformation can then be employed to adjust the mapping's contact angle based on observed values, with the aim of obtaining more precise surface functions for analysing potential variations at the contact angle.


    \item Develop more accurate models for charged droplets controlled by electrodes to optimise the analytical functions. Although the solutions obtained in this study align with experimental observations, the effectiveness of the models has not been directly addressed. Further validation could include exploring whether the dielectric and disc model in a uniform electric field offers a better representation of the droplet surface and complex potential functions, and identifying other charge distribution models that sufficiently account for cases involving electrodes.

\end{itemize}

%The complex potential formed by the charged droplet on a conducting plate, as observed in the conformal mapping study, approximates ellipses with one focus coinciding at a distance, which might have connections to models of galaxy rotation, among others. In fluid dynamics, this phenomenon seems to resemble vortices formed above a depression on the riverbed, which could be explored in future research.


\pagebreak
