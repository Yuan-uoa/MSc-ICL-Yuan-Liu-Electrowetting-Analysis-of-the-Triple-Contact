\section{Summary}
\hspace{0em}\indent From the complex potential function, the surface charge density at the upper edge of the distant slit is derived. The local droplet curve is then formulated as an analytic function of the surface charge density.

Two scenarios were examined: a charged droplet on a plate, where an increase in charge leads to a bulging at the droplet's centre. In the second case of the charge, dielectric, and droplet problem, the derived surface charge density revealed a U-shaped pattern, signifying charge accumulation at the droplet's ends. Furthermore, as the charge magnitude increases, a distinctive double-peak feature emerges on the surface. These findings align closely with physical and numerical studies.

The conformal mapping method effectively describes the properties of charged droplets, but there are limitations to the current approach. The specific mapping function leads to an undefined charge density at the contact point. This appears reasonable, as the contact angle is a physical observation and should not be a property of a specific mapping. Evidently, the mapping method for a distant 0° contact angle in this study cannot address the local contact angle of the droplet.

