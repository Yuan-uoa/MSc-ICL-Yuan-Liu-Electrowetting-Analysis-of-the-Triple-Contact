\chapter{derivations}
\section{mapping}
\subsection{Derivation of $W'(\zeta)$ and $\eta$}\label{cpt:dev_w_e}
\hspace{0em}\indent Derivation of \( W'(\zeta) \)
\[ W'(\zeta) = i\mu \frac{1}{\zeta} \]

Derivation of \( Z'(\zeta) \) as follows:
\begin{equation*}
    \begin{split}
         Z'(\zeta) &= iA \left( -\frac{1}{\zeta^2} + \frac{8(\zeta^2 - a^2) - 16\zeta^2}{(\zeta^2 - a^2)^2} \right)\\
         &= iA \left( -\frac{1}{\zeta^2} - \frac{8(\zeta^2 + a^2)}{(\zeta^2 - a^2)^2} \right)\\
         &= iA \frac{- (\zeta^2 - a^2)^2 - 8\zeta^2 (\zeta^2 + a^2)}{\zeta^2 (\zeta^2 - a^2)^2}\\
         &= iA \frac{-9\zeta^4 - 6a^2\zeta^2 - a^4}{\zeta^2 (\zeta^2 - a^2)^2}\\
         &= \frac{-iA}{\zeta^2 }\frac{(3\zeta^2 + a^2)^2}{(\zeta^2 - a^2)^2}
    \end{split}
\end{equation*}

use the relation:
\[\frac{\df w}{\df z} = \frac{\df W}{\df \zeta}\frac{\df \zeta}{\df z}=\frac{W'(\zeta)}{Z'(\zeta)}  \]

Calculation of \(\eta\)
\[ \eta=\left|\frac{dw}{dz}\right| = \left|\frac{W'(\zeta)}{Z'(\zeta)} \right|=\left|\frac{\mu}{\zeta}\frac{\zeta^2}{A}  \frac{(\zeta^2 - a^2)^2}{(3\zeta^2 + a^2)^2} \right|=\frac{\mu}{A}  \left|\frac{\zeta^2 - a^2}{3\zeta^2 + a^2} \right|^2\]
as $\left|\zeta\right|=1$.



\subsection{try to find the droplet V, fail}\label{cpt:boundary_map_fail}
We expect to find a complex potential, the imaginary part of which is the potential equation of the charged droplet, parameterized by the radii of the concentric circles in the $z$-plane
This mapping transforms points inside the unit circle into points outside the unit circle.
The original conformal mapping is:
\[
\zeta = i A \left( \frac{1}{w} + \frac{8w}{w^2 - a^2} \right)
\]
inverse map:
\[
w = \frac{1}{z}
\]
then, the new composite mapping:
\[
\zeta = i A \left( \frac{1}{\left(\frac{1}{z}\right)} + \frac{8\left(\frac{1}{z}\right)}{\left(\frac{1}{z}\right)^2 - a^2} \right)
\]
Simplify the expression:
\[
\zeta = i A \left( z + \frac{8z}{1 - a^2z^2} \right)
\]
\[
\zeta = i A \left( \frac{9z - a^2z^3}{1 - a^2z^2} \right)
\]

The new composite mapping formula is, in $z(\zeta)$:
\[
z = i A \left( \frac{9\zeta  - a^2\zeta ^3}{1 - a^2\zeta ^2} \right)
\]

With this formula, although the droplet boundary map to the unit circle in the \(\zeta\)-plane correctly, the points inside the semi-circle is not 1-1 mapped to the region out of the droplet curve. 

\subsection{Mapping Points Inside the Unit Circle to the Semi-Circle}\label{cpt:maps_combine}
Given two transformations:

1. Mapping points inside the unit circle to the first quadrant:
   \[
   \zeta_1 = \sqrt{\left( \frac{1 - \zeta}{1 + \zeta} \right) \cdot i}
   \]

2. Mapping points from the first quadrant to the interior of the lower half-circle:
   \[
   \eta_1 = \frac{1 - \eta}{1 + \eta}
   \]

To find the relationship between \(\zeta\) and \(\eta\), set \(\zeta_1 = \eta_1\):
\[
\sqrt{\left( \frac{1 - \zeta}{1 + \zeta} \right) \cdot i} = \frac{1 - \eta}{1 + \eta}
\]

Multiply both sides by \((1 + \eta)\):
\[
\sqrt{\left( \frac{1 - \zeta}{1 + \zeta} \right) \cdot i} \cdot (1 + \eta) = 1 - \eta
\]

Rearrange to isolate \(\eta\):
\[
\sqrt{\left( \frac{1 - \zeta}{1 + \zeta} \right) \cdot i} + \sqrt{\left( \frac{1 - \zeta}{\zeta + 1} \right) \cdot i} \cdot \eta = 1 - \eta
\]

Combine \(\eta\) terms:
\[
\eta \left( \sqrt{\left( \frac{1 - \zeta}{1 + \zeta} \right) \cdot i} + 1 \right) = 1 - \sqrt{\left( \frac{1 - \zeta}{\zeta + 1} \right) \cdot i}
\]

The combination of mappings\footnote{Refer to \ref{cpt:zeta_in_eta} for the inverse form.}, sum up to \(\eta\), is:
\[
\eta = \frac{1 - \sqrt{\left( \frac{1 - \zeta}{\zeta + 1} \right) \cdot i}}{1 + \sqrt{\left( \frac{1 - \zeta}{\zeta + 1} \right) \cdot i}}
\]

\subsection{The Trigs and Hypers}\label{cpt:hyper}
Given Mapping Formula
\[
\eta(\zeta) = \frac{1 - \sqrt{\frac{w}{2}} - i \sqrt{\frac{w}{2}}}{1 + \sqrt{\frac{w}{2}} + i \sqrt{\frac{w}{2}}}
\]

where:
\[
w = \frac{1 - \zeta}{\zeta + 1}
\]

Hyperbolic Function Properties
The definition of the hyperbolic tangent function is:
\[
\tanh(z) = \frac{\sinh(z)}{\cosh(z)} = \frac{e^z - e^{-z}}{e^z + e^{-z}}=\frac{1-e^{-2z}}{1+e^{-2z}}
\]

Transforming the Formula
Considering the complex logarithm:
\[
\log(w) = \log \left( \frac{1 - \zeta}{\zeta + 1} \right)
\]

Let:
\[
u = \sqrt{\frac{w}{2}}
\]

Then the mapping formula can be rewritten as:
\[
\eta(\zeta) = \frac{1 - u - iu}{1 + u + iu}\Longrightarrow-2z=\log {\frac{1+\im}{\sqrt{2}}\sqrt{w}}\Longrightarrow z =-\frac{1}{2}\log{\frac{1+\im}{\sqrt{2}}\sqrt{w}}
\]

Using the Hyperbolic Tangent Function Properties:
\[\eta=
\tanh \left( \frac{1}{2} \log \left( \frac{1+\im}{\sqrt{2}}\sqrt{\frac{1 - \zeta}{\zeta + 1} } \right) \right) = \frac{\sinh \left( \frac{1}{2} \log \left( \frac{1+\im}{\sqrt{2}}\sqrt{\frac{1 - \zeta}{\zeta + 1} } \right) \right) }{\cosh \left( \frac{1}{2} \log \left( \frac{1+\im}{\sqrt{2}}\sqrt{\frac{1 - \zeta}{\zeta + 1} } \right) \right) }
\]

\subsection{inverse map, $\zeta(\eta)$}\label{cpt:zeta_in_eta}
Square both sides to eliminate the square root:

\[
\frac{1 - \zeta}{\zeta + 1} \cdot i = \left( \frac{1 - \eta}{1 + \eta} \right)^2
\]
Expanding and rearranging terms:

\[
\zeta (1 + \eta)^2 - (1 + \eta)^2 = -i \zeta (1 - \eta)^2 - i (1 - \eta)^2
\]

\[
\zeta (1 + 2\eta + \eta^2) - (1 + 2\eta + \eta^2) = -i \zeta (1 - 2\eta + \eta^2) - i (1 - 2\eta + \eta^2)
\]

Combine like terms:

\[
\zeta (1 + 2\eta + \eta^2 + i (1 - 2\eta + \eta^2)) = (1 + 2\eta + \eta^2 - i (1 - 2\eta + \eta^2))
\]

Factor out \(\zeta\):

\[
\zeta (1 + 2\eta + \eta^2 + i (1 - 2\eta + \eta^2)) = (1 + 2\eta + \eta^2 - i (1 - 2\eta + \eta^2))
\]

Finally, solving for \(\zeta\):

\[
\zeta = \frac{(1 - i) + (2 + 2i)\eta + (1 + i)\eta^2}{(1 + i) + (2 - 2i)\eta + (1 + i)\eta^2}
\]
\section{electric studies}
\subsection{The Energy of an Electrostatic Configuration}\label{cpt:energy}
The work required to assemble a distribution of point charges is defined as the energy of the system. Start with a charge Q at the origin, and move a particle of charge $q_1$, from infinity to $\vec{r}_1$. The minimum work of this action is
\begin{equation*}
    W_1 = \int_{\infty}^{r_1}\vec{F}_{Q\,q_1}\cdot\df\vec{l}=-q_1 \int_{\infty}^{r_1}\vec{E}_Q\cdot\df\vec{l}=q_1 V_Q(\vec{r}_1)
\end{equation*}
Next, move a particle of charge $q_2$ from infinity to a distance $\vec{r}_2$ from $Q$, and a distance $\vec{r}_{12}=\vec{r}_2-\vec{r}_1$ from $q_1$. The work done for this action is
\begin{equation*}
    W_2 = \int_{\infty}^{\vec{r}_2}\vec{F}_{Q\,q_2}\cdot\df\vec{l}+\int_{\infty}^{r_{12}}\vec{F}_{q_1\,q_2}\cdot\df\vec{l}=q_2\Big{[}V_Q(\vec{r}_2)+V_{q_1}(\vec{r}_{12})\Big{]}
\end{equation*}
Consequently, the total work is equal to \textbf{half} of the double sum of the products of all interior electric potentials and charges.
\begin{equation*}
    W =\frac{1}{2} \sum_{i=Q,1,2}q_i\sum_{j\neq i} V_j(\vec{r}_i)=\frac{1}{2} \sum_{i=Q,1,2}q_i V(\vec{r}_{i})\hspace{0.5em},\hspace{1em}V(r_i)\coloneqq\sum_{j\neq i} V_j(\vec{r}_{ij})
\end{equation*}
In the case of a continuous charge distribution, the energy of the static charge system in free space is 
\begin{equation*}
    W\footnote{this is equation (2.43) of \cite{Griffiths_2017}, page 94.}=\frac{1}{2}\int_{\mathcal{V}}\rho\, \int \vec{E} \cdot \df \vec{l} \,\df \tau=\frac{1}{2}\int_{\mathcal{V}}\rho\,V \,\df \tau
\end{equation*}
rewriting using equation (\ref{E.rho}), and applying the divergence theorem
\begin{equation*}
\begin{aligned}
    W&=\frac{1}{2}\int_{\mathcal{V}}\epsilon_0(\nabla\cdot \vec{E})\, V\df \tau\\
    \frac{\epsilon_0}{2}\times&=\int_{\mathcal{V}}\nabla\cdot(\vec{E} V)\df \tau-\int_{\mathcal{V}}\vec{E}\cdot
       \eqnmarkbox[black]{node1}{(\nabla V)} \df\tau\\
    &\sim \left.\underbrace{\oint_{S}\vec{E}V\df \vec{a}}_{\sim r^{-1}\sim0}+\int_\mathcal{V}E^2\df\tau\right|_{\vec{r}\rightarrow\infty}^{\text{let integral over all space}}
\end{aligned}
\end{equation*}
\annotate[yshift=1em]{above, right, label above}{node1}{$\nabla V= -\vec{E}$}
All of the above leads to the energy of/(work to assemble) an electrostatic configuration:
\begin{equation}
    W=\frac{\epsilon_0}{2}\int E^2 \df \tau\hspace{0.5em},\hspace{1em}\text{(all space)}
\end{equation}
\iffalse
A capacitor is two conductors separated by a dielectric or free space. Define \textbf{capacitance} $C\coloneqq \frac{Q}{V}$.
For the free space case, the energy (of a capacitor) is:
\[
W=\frac{1}{2}CV^2
\]
for dielectric,
\[
W=\frac{1}{2}\epsilon_r C V^2=\frac{\epsilon_r\,\epsilon_0}{2}\int E^2 \df \tau\hspace{0.5em},\hspace{1em}\text{(all space)}  
\]
\fi
\subsection{Proof: Uniqueness of Electric Potential}\label{cpt:uniqueness}
\begin{proof}
Let $V^{(1)}$ and $V^{(2)}$
be two random electric potential fields of the scenario. Define
\[V\coloneqq V^{(1)} - V^{(2)}\]
\[\vec{E}\coloneqq\vec{E}^{(1)} - \vec{E}^{(2)}\]
Assume that at the boundary $s\equiv z\to\infty$
\[V^{(1)}\big|_{s} =V^{(2)}\big|_{s}\Longrightarrow V(z\to\infty) =V\big|_{s}= 0\]
apply the divergence theorem, at $z\to\infty$:
\[\int_\mathcal{V} \nabla \cdot (V \vec{E}) \, \df\tau= \int_S V \vec{E} \cdot \hat{n} \, \df a = 0\]
take the divergence of $V\vec{E}$:
\[\nabla \cdot (V \vec{E}) = \nabla V \cdot \vec{E}+ V \cdot (\nabla\cdot\vec{E}) = - \vec{E}\cdot \vec{E}\]
from the two we find
\[\int_{\mathcal{V}} - \vec{E}^2 \, dV = 0\]
therefore $\vec{E}\cdot\vec{E}\equiv 0\text{ }\forall z$, equivalently $E_x^2+E_y^2\equiv 0\hspace{0.5em}\forall\, z$, hence $E_x\equiv 0,\,E_y\equiv 0\hspace{0.5em}\forall\, z$, implies
\[\vec{E^{(1)}}=\vec{E^{(2)}}\hspace{0.5em}\forall\, z\]
Finding this, together with the two voltages being equal at the boundary $z\to\infty$, shows that the values of the two electric potentials are identical. That is
\[V^{(1)} \big|_{s} = V^{(2)}\big|_{s}\Longrightarrow V^{(1)} = V^{(2)} \hspace{0.5em} \forall\, z\]
\end{proof}
\subsection{Uncharged Conductor Disc with Constant E Apply}
Exist constant $E_0$, let
\[
    V_{far}=E_0 r \sin{\theta}
\]
with Laplace slon \[V = \sum (A_n r^n+B_n) \frac{1}{r^n}(\sin{n\theta}+\cos{n\theta})\]
may find
\[
    V=k_0 (r-\frac{1}{r})\sin{\theta}
\]
and surface charge at r=1
\[
\sigma=-\epsilon_0\frac{\partial V}{\partial r}=-\epsilon_0 (1+\frac{1}{r^2})\sin{\theta}=-2k_0\epsilon_0\sin{\theta}
\]

\subsection{Potentials, vortex case}
\[w=\log (z -b) + \log (z-\frac{a^2}{b})-\log(z)\Longrightarrow w =\log (z +\frac{a^2}{z}-b-\frac{a^2}{b})\]
this function\footnote{Refering to Fluid Dynamic 1 exam paper year 2014 question 4.} is a combination of one source at $b$, another one at $\frac{a^2}{b}$, and a sink at $0$.

on $z=a e^{\mathrm{i}  \theta}$
\[ w=\log (2 a \cos{\theta}-b-\frac{a^2}{b})\]
hence
\[\Im[w]= 0 \text{ or } -\pi\]

%\renewcommand\bibname{{References}}
%\bibliography{References}
%\bibliographystyle{plain}