\chapter{Preliminaries}
This section presents and derives the fundamental setup of the study. 

\section{Complex Analysis}


\section{Electrostatics}
This section is based on Chapters 2, 3, and 4 of \cite{Griffiths_2017} and briefly explains the electrostatics relevant to this study.

\subsection{Point Charge}
According to Coulomb's Law, the force $\mathbf{F}_{Q\,q}$ (in Newtons) generated by a particle of charge $Q$ (in Coulombs) at the origin, on a test particle of charge $q$ (in Coulombs) at a distance $\mathbf{r}$ (in meters) is: 
\begin{equation*}
    \mathbf{F}_{Q\,q} = \frac{1}{4\pi\,r^2}\frac{Q\,q}{\epsilon_0}\mathbf{\hat{r}}\hspace{0.5em},\hspace{1em}    \epsilon_0=8.85\times 10^{-12} \frac{C^2}{N\cdot m^2}
\end{equation*}
The constant $\epsilon_0$ is the \textbf{permittivity of free space}.
$\mathbf{E}_Q$, the electric field of a charge $Q$,  is
\begin{equation*}
    \mathbf{E}_Q (\mathbf{r})\coloneqq \frac{\mathbf{F}_{Q\,q}}{q}=\frac{1}{4\pi\,r^2}\frac{Q}{\epsilon_0}\mathbf{\hat{r}}
\end{equation*}
In the context of continuous charge distribution in space, the closed surface integral of the electric field $E$ is:
\begin{equation*}
    \oint_S \mathbf{E}\cdot\df\mathbf{a}=\frac{Q_{enclose}}{\epsilon_0}
\end{equation*}
$Q_{enclose}$ is the charge enclosed by the closed surface.\\
By applying the divergence theorem
\begin{equation}\label{E.rho}
    \int_{\mathcal{V}}\nabla\cdot \mathbf{E} \df \tau=\frac{Q_{enclose}}{\epsilon_0} \Longrightarrow \nabla\cdot\mathbf{E}=\frac{\rho}{\epsilon_0}\hspace{0.5em},\hspace{1em}\rho\coloneqq\left.\frac{Q_{enclose}}{V}\right|_{r\rightarrow 0}
\end{equation}

The electric potential of a charge Q is: 
\begin{equation*}
    V_Q(\mathbf{r}_1)\coloneqq-\int_{\infty}^{r_1}\mathbf{E}_Q\cdot\df\mathbf{l}=\frac{1}{4\pi r_1}\frac{Q}{\epsilon_0}
\end{equation*}
Assume $V(\mathbf{r}\rightarrow\infty)=0$.

\subsection{Conductor}

Charges are free to move and distributed only on the surface of a conductor to minimise its potential energy. The electric field is zero inside a conductor and perpendicular to the boundary surface of the conductor.\footnote{Refer to \cite{Griffiths_2017}, pp. 99}\\
On the boundary line of area $\Omega\in\mathbb{R}^2$, the electric fields are correspondingly\footnote{Refer to \cite{Griffiths_2017}, pp. 88-90, 103-104.}
\begin{equation}\label{ed.line}
\mathbf{E}|_{\partial \Omega}=\frac{\lambda}{\epsilon_0}\hat{n}
\end{equation}
Hence, the force per unit boundary line of a conductor is
\[\mathbf{f}=\frac{1}{2\epsilon_0}\lambda^2\hat{\mathbf{n}}\]
and the \textbf{electrostatic pressure} on the boundary is
\[P\coloneqq\frac{\epsilon_0}{2}E^2=\frac{1}{\epsilon}\left(\frac{\lambda}{\epsilon_0}\right)^2\]



\subsection{The Case of Dielectrics}
Dielectrics is the synonym for insulator, the opposite of conductor, in which charges are not free to move within the material, but attached to specific structures such as atoms. As a consequence, the relationship between the electric field inside a linear dielectric which surrounded a particle with free charge Q is affected by multiplying a dimensionless quantity $\epsilon_r$, the relative permittivity
\[\mathbf{E}=\frac{1}{2\pi r\epsilon_0}\frac{Q}{\epsilon_r}\hat{\mathbf{r}}\]
the possibly induced surface charge and volume charge are
\[\sigma_b\coloneqq\mathbf{P}\cdot\hat{n}\]
and
\[\rho_b\coloneqq-\nabla\cdot\mathbf{P}\]
\[\mathbf{P}=\epsilon_0\chi_e\vec{E}\]
where
\[
\mathbf{D}=\epsilon \vec{E}
\]
\[
\epsilon=\epsilon_0\epsilon_r
\]
\[
\epsilon_r=1+\chi_e
\]

\subsection{The Energy of an Electrostatic Configuration}
The work required to assemble a distribution of point charges is defined as the energy of the collection of charges.

\noindent Start by moving charge Q to the origin. No work is required since there is no electric field and hence no force. After that, move a particle of charge $q_1$, from infinity to $\mathbf{r}_1$. The minimum work of the activity is

\begin{equation*}
    W_1 = \int_{\infty}^{r_1}\mathbf{F}_{Q\,q_1}\cdot\df\mathbf{l}=-q_1 \int_{\infty}^{r_1}\mathbf{E}_Q\cdot\df\mathbf{l}=q_1 V_Q(\mathbf{r}_1)
\end{equation*}

\noindent Next, move a particle of charge $q_2$ from infinity to a distance $\mathbf{r}_2$ from $Q$, correspondingly a distance $\mathbf{r}_{12}=\mathbf{r}_2-\mathbf{r}_1$ from $q_1$. The work is

\begin{equation*}
    W_2 = \int_{\infty}^{\mathbf{r}_2}\mathbf{F}_{Q\,q_2}\cdot\df\mathbf{l}+\int_{\infty}^{r_{12}}\mathbf{F}_{q_1\,q_2}\cdot\df\mathbf{l}=q_2\Big{[}V_Q(\mathbf{r}_2)+V_{q_1}(\mathbf{r}_{12})\Big{]}
\end{equation*}

Consequently, the whole work is equal to \textbf{half} of the double sum of the products of all interior electric potentials and charges.

\begin{equation*}
    W =\frac{1}{2} \sum_{i=Q,1,2}q_i\sum_{j\neq i} V_j(\mathbf{r}_i)=\frac{1}{2} \sum_{i=Q,1,2}q_i V(\mathbf{r}_{i})\hspace{0.5em},\hspace{1em}V(r_i)\coloneqq\sum_{j\neq i} V_j(\mathbf{r}_{ij})
\end{equation*}

In the case of a continuous charge distribution, the energy of the static charge system in free space is 

\begin{equation*}
    W\footnote{this is (2.43) of \cite{Griffiths_2017}, pp.94}=\frac{1}{2}\int_{\mathcal{V}}\rho\, \int \mathbf{E} \cdot \df \mathbf{l} \,\df \tau=\frac{1}{2}\int_{\mathcal{V}}\rho\,V \,\df \tau
\end{equation*}

rewriting using equation (\ref{E.rho}), and applying the divergence theorem

\begin{equation*}
\begin{aligned}
    W&=\frac{1}{2}\int_{\mathcal{V}}\epsilon_0(\nabla\cdot \mathbf{E})\, V\df \tau\\
    \frac{\epsilon_0}{2}\times&=\int_{\mathcal{V}}\nabla\cdot(\mathbf{E} V)\df \tau-\int_{\mathcal{V}}\mathbf{E}\cdot
       \eqnmarkbox[black]{node1}{(\nabla V)} \df\tau\\
    &\sim \left.\underbrace{\oint_{S}\mathbf{E}V\df \mathbf{a}}_{\sim r^{-1}\sim0}+\int_\mathcal{V}E^2\df\tau\right|_{\mathbf{r}\rightarrow\infty}^{\text{let integral over all space}}
\end{aligned}
\end{equation*}
\annotate[yshift=1em]{above, right, label above}{node1}{$\nabla V= -\mathbf{E}$}
All of the above leads to the energy of/(work to assemble) an electrostatic configuration:
\begin{equation}
    W=\frac{\epsilon_0}{2}\int E^2 \df \tau\hspace{0.5em},\hspace{1em}\text{(all space)}
\end{equation}

A capacitor is two conductors separated by a dielectric or free space. Define \textbf{capacitance} $C\coloneqq \frac{Q}{V}$.
For the free space case, the energy (of a capacitor) is:
\[
W=\frac{1}{2}CV^2
\]
for dielectric,
\[
W=\frac{1}{2}\epsilon_r C V^2=\frac{\epsilon_r\,\epsilon_0}{2}\int E^2 \df \tau\hspace{0.5em},\hspace{1em}\text{(all space)}  
\]

\subsection{Uniqueness of Electrostatic Potential}\label{uniqueness}
specify far field condition to find a unique soln. 

The electric potential is uniquely decided\footnote{There is only one stokes flow for a specified $\mathbf{u}$ on the boundary.}. Let $V^{(1)}$ and $V^{(2)}$
be two random electric potential fields of the scenario. Define
\[V\coloneqq V^{(1)} - V^{(2)}\]
\[\vec{E}\coloneqq\vec{E}^{(1)} - \vec{E}^{(2)}\]
it is reasonable to assume that at the boundary $s\equiv z\to\infty$
\[V^{(1)}\big|_{s} =V^{(2)}\big|_{s}\]
hence
\[V(z\to\infty) =V\big|_{s}= 0\]
apply the divergence theorem, at $z\to\infty$:
\[\int_\mathcal{V} \nabla \cdot (V \vec{E}) \, \df\tau= \int_S V \vec{E} \cdot \hat{n} \, \df a = 0\]
take divergence of $V\vec{E}$:
\[\nabla \cdot (V \vec{E}) = \nabla V \cdot \vec{E}+ V \cdot (\nabla\cdot\vec{E}) = - \vec{E}\cdot \vec{E}\]
from the two we find
\[\int_{\mathcal{V}} - \vec{E}^2 \, dV = 0\]
therefore $\vec{E}\cdot\vec{E}\equiv 0\text{ }\forall z$, equivalently $E_x^2+E_y^2\equiv 0\hspace{0.5em}\forall\, z$, hence $E_x\equiv 0,\,E_y\equiv 0\hspace{0.5em}\forall\, z$, implies
\[\vec{E^{(1)}}=\vec{E^{(2)}}\hspace{0.5em}\forall\, z\]

Finding this, together with the two voltages being equal at the boundary $z\to\infty$, shows that the values of the two electric potentials are identical. That is
\[V^{(1)} \big|_{s} = V^{(2)}\big|_{s}\Longrightarrow V^{(1)} = V^{(2)} \hspace{0.5em} \forall\, z\]

Since the two electric potentials are equal everywhere, we claim that there is a unique electric potential and complex potential, corresponding to a specified voltage on a boundary.
\section{Contact Angle and related}\footnote{This part is summarized form \cite{Fontelos2008}, with some supplement according to the study.}

Consider a homogeneous and smooth plane of solid material in contact with air. The surface energy within contact area \(A\) is: \[E_1=\gamma_{sv}\, A\]
$\gamma\coloneqq\frac{F}{L}$ is the \textbf{surface tension}. In $\mathbf{R}^3$,  $[\gamma]=\frac{N}{m}$ and $[\gamma A]=J$ .\\
\indent When a liquid droplet occupies the same area \(A\) in contact with the solid, and contacts the gas over an area \(A_{vl}\), the corresponding energy is
\[E_2=\gamma_{sl} A + \gamma_{vl} A_{vl}\]
Where $\gamma_{sv}$, $\gamma_{sl}$, $\gamma_{sv}$ are the surface tensions of the solid/vapour, solid/liquid, and vapour/liquid interfaces. For simplicity, the hysteresis in this relationship is disregarded.\\
\indent For a small enough droplet, gravitation is negligible. The droplet will stabilize if the energy difference is minimized\footnote{This seemingly self-evident principle is said to be discussed in \cite{Gibbs1878}. \color{red}check later}, rather than minimize $E_2$.
\[\Delta E = E_2-E_1=(\gamma_{sl}-\gamma_{sv})A+\gamma_{vl}A_{vl}\]
\indent The boundary of the droplet tends to be smaller, while the volume of the droplet is a constant. 
A special situation is the complete wetting case, when $A=A_{vl}$, and the liquid will cover the solid surface as much as possible. 
\[\Delta E = E_2-E_1=[(\gamma_{sl}-\gamma_{sv})+\gamma_{vl}]A\]
\indent The \textbf{contact angle} is the exterior angle the drop makes with the material surface, indicating the hydrophilicity of the material. Consider the edge of the contact surface between a droplet and a solid plane, when the droplet is stabilized, the resultant force is zero. Thus, in the direction along the plane, we have Young's Relation equation:
\[\gamma_{vl} \cos \theta_{eq} = \gamma_{sv} - \gamma_{sl}\]
and the contact angle $\theta_{eq}$
\[
\cos \theta_{eq}=\frac{\gamma_{sv}-\gamma_{sl}}{\gamma_{vl}}
\]
When an external voltage is applied\\
...

\section{Curvature and Related}
\subsection{Curvature and Normal Stress}
\indent The curvature measures the normal displacement as the surface or line extends, and is defined as the divergence of the normal direction of a curve \( f(\mathbf{r}) \):
\[ K \coloneqq \nabla \cdot \frac{\nabla f}{|\nabla f|} \]
For a length extension \(\Delta L\) on the curve, the increase in the normal direction is \( K \Delta L \).\\
\indent Consider a material (fluid or solid) with a continuous and curved boundary surface in $\mathbb{R}^3$ or a curved boundary line in $\mathbb{R}^2$. The \textbf{normal stress per unit length} due to the curvature, that is the force in the normal direction along the curve on unit length due to surface tension, is:
\[
\mathbf{F}_{\gamma}=\gamma K\hat{\mathbf{n}}
\]
\section{Bernoulli Condition}
\subsection{Bernoulli Condition with curvature}
Bernoulli Condition is
\[
P+\frac{1}{2}\rho v^2+\rho g h = Cst\hspace{0.5em},\hspace{1em}\text{on a streamline}
\]
With curvature, the pressure difference at points on the streamline is $\Delta p =\gamma K$. Let $g=0$, and pick point 1 as a reference and random point 2 on the streamline, 
\[\frac{P_1}{\rho}+\frac{1}{2}v_1^2=\frac{P_2}{\rho}+\frac{1}{2}v_2^2=Cst\Longrightarrow \frac{\gamma K}{\rho}+\frac{1}{2}v_2^2=\frac{1}{2}v_1^2 =Cst_1\]
This corresponds to the non-dimensionalised form in equation (10) of  \cite{Crowdy2000} and equation (3) of \cite{Crowdy1999}
\[
\kappa+\frac{1}{2}\left|\frac{\df w(z)}{\df z}\right|^2=Cst
\]

%\pagebreak
%\renewcommand\bibname{{References}}
%\bibliography{References}
%\bibliographystyle{plain}