\section{Electrostatics}
\subsection{Point Charge}
According to Coulomb's Law, the force $\vec{F}_{Q\,q}$ generated by a particle of charge $Q$ at the origin$\in\mathbb{R}^3$, on a particle of charge $q$ at a distance $\vec{r}$ is: 
\begin{equation*}
    \vec{F}_{Q\,q} = \frac{1}{4\pi\,r^2}\frac{Q\,q}{\epsilon_0}\hat{r}\hspace{0.5em},\hspace{1em}    \epsilon_0=8.85\times 10^{-12} \frac{C^2}{N\cdot m^2}
\end{equation*}
The constant $\epsilon_0$ is the permittivity of free space.
$\vec{E}_Q$, the electric field of a charge $Q$,  is\vspace{-0.5em}
\begin{equation*}
    \vec{E}_Q (\vec{r})\coloneqq \frac{\vec{F}_{Q\,q}}{q}=\frac{1}{4\pi\,r^2}\frac{Q}{\epsilon_0}\hat{r}\vspace{-1.em}
\end{equation*}
In the context of continuous charge distribution in space, Gauss's Law describes the closed surface integral of the electric field $E$:
\vspace{-1.5em}
\begin{equation*}
    \oint_S \vec{E}\cdot\df\vec{a}=\frac{Q_{enclose}}{\epsilon_0}
\end{equation*}
$Q_{enclose}$ is the charge enclosed by the closed surface. By applying the divergence theorem
\begin{equation}\label{E.rho}
    \int_{\mathcal{V}}\nabla\cdot \vec{E} \df \tau=\frac{Q_{enclose}}{\epsilon_0} \Longrightarrow \nabla\cdot\vec{E}=\frac{\rho}{\epsilon_0}\hspace{0.5em},\hspace{1em}\rho\coloneqq\left.\frac{Q_{enclose}}{Volume}\right|_{r\rightarrow 0}
\end{equation}

\noindent Assume $V(\vec{r}\rightarrow\infty)=0$, The electric potential of a charge Q is: 
\begin{equation*}
    V_Q(\vec{r}_1)\coloneqq-\int_{\infty}^{r_1}\vec{E}_Q\cdot\df\vec{l}=\frac{1}{4\pi r_1}\frac{Q}{\epsilon_0}
\end{equation*}

The energy of/(work to assemble) an electrostatic configuration\footnote{Refer to (\ref{cpt:energy}) for detailed derivation}:
\begin{equation}
    W=\frac{\epsilon_0}{2}\int E^2 \df \tau\hspace{0.5em},\hspace{1em}\text{(all space)}
\end{equation}
\subsection{Conductor}
\label{cpt:Conductor}
\hspace{0em}\indent Charges are distributed only on the surface of a conductor to minimise its electric potential energy. The electric field is zero inside a conductor and perpendicular to the boundary surface/line of the conductor\footnote{Refer to \citet{Griffiths_2017}, page 99.}, to ensure that electrons resting on the conductor's surface hence no current is generated. The electric fields are correspondingly\footnote{Refer to \citet{Griffiths_2017}, page 88-90, 103-104.}
\begin{equation}\label{eqn:ed.line}
\vec{E}|_{\partial \Omega}=\frac{\sigma}{\epsilon_0}\hat{n}
\end{equation}
Hence, the electrostatic pressure is\footnote{This refers to \citet{Griffiths_2017}, page 103. Only the outside of the conductor has $\vec{E}\neq 0$, hence there is a $\frac{1}{2}$.}
\begin{equation}\label{eqn:f.line}
    p_e\coloneqq\frac{\epsilon_0}{2}E^2=|\vec{f}_e|=\frac{\sigma^2 }{2\epsilon_0}
\end{equation}
$f_e$ is the force per unit boundary line of a conductor.

\subsection{Dielectrics}
Dielectric is the synonym for insulator, the opposite of conductor, in which charges are not free to move but attached to structures such as atoms. As a consequence, the relationship between the electric field inside a linear dielectric which surrounded a particle with free charge Q is affected by dimensionless quantity $\epsilon_r$, the relative permittivity
\[\vec{E}=\frac{1}{2\pi r\epsilon_0}\frac{Q}{\epsilon_r}\hat{r}\]
\begin{equation}\label{eqn:P}
    \vec{P}\coloneqq\epsilon_0\chi_e\vec{E}    
\end{equation}
where
% \[
% \vec{D}\coloneqq\epsilon \vec{E}
% \]
\[
\epsilon\coloneqq\epsilon_0\epsilon_r
\]
\[
\epsilon_r=1+\chi_e
\]
the induced surface charge and volume charge are
\begin{equation}\label{eqn:die surf q}
    \sigma_b\coloneqq\vec{P}\cdot\hat{n}    
\end{equation}
\begin{equation}\label{eqn:rho_b}
    \rho_b\coloneqq-\nabla\cdot\vec{P}    
\end{equation}
\subsection{Uniqueness of Electric Potential}
\begin{thm}\label{thm:uniquenss}
    Specifying the voltage on a closed boundary determines a unique potential\footnote{Recall: There is only one stokes flow for a specified $\vec{u}$ on the boundary.}.
\end{thm}
for proof see \ref{cpt:uniqueness}.
