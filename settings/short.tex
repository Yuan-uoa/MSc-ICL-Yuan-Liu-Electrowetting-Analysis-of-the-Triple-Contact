%\newcommand{\apx}{\, \~{} \,}
\newcommand{\im}{\mathrm{i} \,}
\newcommand{\df}{\,\mathrm{d}}%定义命令来表示微分符号



\newtheorem{thm}{Theorem}[section]%
%\newtheorem{eg}{E.g.}[section]%
%\newtheorem{defn}{Definition}[section]%
\newtheorem{defn}[thm]{Definition}% def 不工作,需定义新命令 defn
\newtheorem{eg}[thm]{Example}%
%替换的两条,使 eg thm def 统一编号,便于查找。
\newtheorem{prop}[thm]{Propostion}%
\newtheorem{prf}[thm]{Proof}%


\def\Xint#1{\mathchoice
   {\XXint\displaystyle\textstyle{#1}}%
   {\XXint\textstyle\scriptstyle{#1}}%
   {\XXint\scriptstyle\scriptscriptstyle{#1}}%
   {\XXint\scriptscriptstyle\scriptscriptstyle{#1}}%
   \!\int}
\def\XXint#1#2#3{{\setbox0=\hbox{$#1{#2#3}{\int}$}
     \vcenter{\hbox{$#2#3$}}\kern-.5\wd0}}
\def\ddashint{\Xint=}
\def\dashint{\Xint-}
%\dashint gives a single-dashed integral sign, \ddashint a double-dashed one.
%https://texfaq.org/FAQ-prinvalint