\usepackage{xeCJK}%汉字

%\usepackage[dvipsnames,table]{xcolor}%多颜色,颜色库要放在最前边,否则可能影响载入。
\usepackage[dvipsnames]{xcolor}%多颜色,颜色库要放在最前边,否则可能影响载入。

%\usepackage[options]{package-name}
%多个option不知道是不是用,隔开

%\usepackage[table]{xcolor}% to color the table

\usepackage{mdframed}%加灰色底框

\usepackage{empheq}%给文本加框

\usepackage[perpage]{footmisc}%footnote re-no. every page

%\usepackage{lipsum}

%\usepackage[utf8]{inputenc}
\usepackage{amsmath,amsthm,amssymb}%美国数学协会,proof 命令在这里。不是第一个
% 定义新的定理样式
\newtheoremstyle{mytheoremstyle}% 请替换为你喜欢的样式名
  {0.5em}% 上间距
  {0.5em}% 下间距
  {\itshape}% 定理内容的字体样式
  {}% 缩进量
  {\bfseries}% 定理标题的字体样式
  {  }% 标题后的标点符号
  {0.0em}% 标题与内容之间的距离
  {}% 定理头部的额外规范
\theoremstyle{mytheoremstyle}% 创建新的定理环境
\newtheorem{theorem}{Theorem}

\usepackage{float}%调整图片位置的包,才能用[H]
\usepackage{adjustbox}%供画图用
\definecolor{mycolor}{RGB}{82, 140, 164}%调好的蓝色,做边框
%E.G.一个画图例子
% begin{figure}[H] % [H] forces figure to be output where it is defined in code (it suppresses floating)
% 		\adjustbox{frame=1pt,frame,margin=0.5,color=mycolor}{\includegraphics[width=\linewidth]{pics/The Nectar login page.png}} % Figure image
% 		\caption{\small Regenerated boundary shape and surface charge density of the charged droplets as described by Crowdy (2015), with different value of capillarity effort coefficients applied.}
% 	\end{figure}


\usepackage{extarrows}%长等号,且上下可添加文字
%https://www.latexstudio.net/archives/8004.html
%$$ A\xlongequal[sub-script]{super-script}B $$


%链接:https://www.zhihu.com/question/617310608/answer/3166157888
%\usepackage{amssymb}%可以打\therefore 因为所以符号的命令

\usepackage{mathrsfs}%花体字母

\numberwithin{equation}{section}%和上一个,让公式按 section 编号

\usepackage{bm}
%\usepackage{natbib}

%%%%%%%%%%%%%%%%%%%%%%%%%%%%%%%%%%%%%%%%%%%%%%%%%%%%%%%%
%annotate eqn packages
\usepackage{tikz}
\usetikzlibrary{backgrounds}
\usetikzlibrary{arrows,shapes}
\usetikzlibrary{tikzmark}
\usetikzlibrary{calc}

\usepackage{mathtools}
\usepackage{hyperref}
\usepackage{cleveref}
\usepackage{annotate-equations}
%%%%%%%%%%%%%%%%%%%%%%%%%%%%%%%%%%%%%%%%%%%%%%%%%%%%%%%%%%%%%%%%%%%%%%%%%%%

%\usepackage{float}%禁止表格浮动,相应的命令是\FloatBarrier,但实际上似乎不管用,用[h!]解决的。


%\usepackage[a4paper,scale=0.8]{geometry}%格式

%\usepackage{color}


%\usepackage{lipsum}

\usepackage{ragged2e}%使用\justify 回到正常对齐

\usepackage{tabularray}

\usepackage{shadowtext}

%\usepackage[utf8]{inputenc}
\usepackage[T1]{fontenc}
\usepackage{lmodern}
\usepackage{amsfonts}
\usepackage{epstopdf}
%\usepackage{matlab}

\sloppy%不知道干嘛用,matlab的文件带的
\epstopdfsetup{outdir=./}
\graphicspath{ {./a1q3_images/} }

\usepackage{pdfpages}%嵌入pdf页面,用\includepdf[pages={1,2}]{1.pdf} %[pages={3},scale=0.5] 命令可用

%公式字母划去,用\cancel{}
\usepackage{cancel}
%\usepackage[thicklines]{cancel}

\usepackage{graphicx} % Required for inserting images


%以下三个是griffiths用的表示电磁学上场点和源点差的矢量的符号,需要相应的两个pdf的支持。
%\def\rcurs{{\mbox{$\resizebox{.16in}{.08in}{\includegraphics{ScriptR}}$}}}
%\def\brcurs{{\mbox{$\resizebox{.16in}{.08in}{\includegraphics{BoldR}}$}}}
%\def\hrcurs{{\mbox{$\hat \brcurs$}}}

%%%%%%%%%%%%%%%%%%%%%%%%%%%%%%%%%%%%%%%%%%%%%%%%%%%%%%%%%%%%%%%%%%%%%%%%%%
\usepackage{fancyhdr}
\pagestyle{fancy}
\usepackage{lastpage}

% Define Header
%\lhead{Yuan Liu (313350417)}
%\chead{Physics 335 QM formulas}
%\rhead{\today}

% Define Footer
%\lfoot{}
%\cfoot{Page \thepage \,of \pageref{LastPage}}
%\rfoot{}

%\fancyhead{} % 页眉清空
\renewcommand{\headrulewidth}{0pt} % 去页眉线
%Define width of horizontal line in the header
%\renewcommand{\headrulewidth}{0.4pt}
\renewcommand{\footrulewidth}{0.2pt}